\chapter*{Введение}
\addcontentsline{toc}{chapter}{ВВЕДЕНИЕ}

В современном образовании одним из ключевых моментов является определение выпускного балла студентов, который отражает их успеваемость и готовность к завершающему этапу обучения. Однако, при анализе факторов, влияющих на формирование этого показателя, становится ясно, что образ жизни студентов играет существенную роль.

Исследование данных об образе жизни студентов может дать ценную информацию о их поведении, привычках, уровне активности и здоровье, которые в свою очередь могут оказывать влияние на их успеваемость и результаты обучения. Факторы, такие как режим дня, питание, физическая активность, уровень стресса и т.д., могут быть важными при определении выпускного балла студентов.

Таким образом, анализ данных об образе жизни студентов может помочь выявить взаимосвязи между их поведением и успехами в учебе, что в свою очередь может быть использовано для более точного определения критериев успешности обучения и разработки эффективных стратегий поддержки студентов. В данном контексте, использование методов анализа данных и моделей регрессии может стать мощным инструментом для выявления закономерностей и прогнозирования выпускного балла студентов на основе данных об их образе жизни.

Цель работы -- решить задачу регрессии, состоящую в определении выпускного балла студентов с помощью информации об их образе жизни, на данных о студентах двух португальских школ, взятых с kaggle \cite{bib:kaggle}.

Для достижения поставленной цели требуется решить следующие задачи:

\begin{itemize}
    \item описать набор данных и визуализировать его;
    \item проанализировать существующие модели регрессии для решения задачи и выбрать наиболее подходящую, а также выбрать функционал качества модели;
    \item описать алгоритм предобработки данных;
    \item описать общий алгоритм работы ПО, осуществляющего решение задачи регрессии;
    \item разработать ПО, решающее задачу регрессии;
    \item выбрать гиперпараметры модели, с которыми модель работает наилучшим образом с точки зрения выбранного функционала качества, оценить обобщающую способность;
    \item провести исследование ПО с целью сравнения полученной модели с другими моделями.
\end{itemize}

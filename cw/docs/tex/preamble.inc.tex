\documentclass[a4paper,14pt]{extreport}

\usepackage{cmap} % Улучшенный поиск русских слов в полученном pdf-файле
\usepackage[T2A]{fontenc} % Поддержка русских букв
\usepackage[utf8]{inputenc} % Кодировка utf8
\usepackage[english,russian]{babel} % Языки: русский, английский
%\usepackage{pscyr} % Нормальные шрифты
\usepackage{amsmath}
\usepackage{geometry}
\geometry{left=30mm}
\geometry{right=10mm}
\geometry{top=20mm}
\geometry{bottom=20mm}
\usepackage{titlesec}
%\usepackage[ruled,longend]{algorithm2e}
\usepackage{textgreek}
\usepackage{algpseudocode}
\usepackage{pgfplots}
\pgfplotsset{compat=1.9}
\usepackage{pdfpages}
\usepackage{algorithm}
\renewcommand{\listalgorithmname}{Список алгоритмов}
\floatname{algorithm}{Алгоритм}
\usepackage{amssymb}
\titleformat{\section}
{\normalsize\bfseries}
{\thesection}
{1em}{}
\titlespacing*{\chapter}{0pt}{-30pt}{8pt}
\titlespacing*{\section}{\parindent}{*4}{*4}
\titlespacing*{\subsection}{\parindent}{*4}{*4}
\usepackage{setspace}
\usepackage{mathtools}
\usepackage{float}
\DeclarePairedDelimiter\bra{\langle}{\rvert}
\DeclarePairedDelimiter\ket{\lvert}{\rangle}
\DeclarePairedDelimiterX\braket[2]{\langle}{\rangle}{#1 \delimsize\vert #2}
\onehalfspacing % Полуторный интервал
\frenchspacing
\usepackage{indentfirst} % Красная строка
\usepackage{titlesec}
\titleformat{\chapter}{\LARGE\bfseries}{\thechapter}{20pt}{\LARGE\bfseries}
\titleformat{\section}{\Large\bfseries}{\thesection}{20pt}{\Large\bfseries}
\usepackage{listings}
\usepackage{xcolor}

\usepackage{pgfplots}
\usetikzlibrary{datavisualization}
\usetikzlibrary{datavisualization.formats.functions}
\usepackage{graphicx}
\newcommand{\img}[3] {
    \begin{figure}[h]
        \center{\includegraphics[height=#1]{assets/img/#2}}
        \caption{#3}
        \label{img:#2}
    \end{figure}
}
\newcommand{\boximg}[3] {
    \begin{figure}[h]
        \center{\fbox{\includegraphics[height=#1]{assets/img/#2}}}
        \caption{#3}
        \label{img:#2}
    \end{figure}
}
\usepackage[justification=centering]{caption} % Настройка подписей float объектов
\usepackage[unicode,pdftex]{hyperref} % Ссылки в pdf
\hypersetup{hidelinks}
\newcommand{\code}[1]{\texttt{#1}}
\usepackage{icomma} % Интеллектуальные запятые для десятичных чисел
\usepackage{csvsimple}

\usepackage{color} %use color
\definecolor{mygreen}{rgb}{0,0.6,0}
\definecolor{mygray}{rgb}{0.5,0.5,0.5}
\definecolor{mymauve}{rgb}{0.58,0,0.82}

%Customize a bit the look
\lstset{ %
	backgroundcolor=\color{white}, % choose the background color; you must add \usepackage{color} or \usepackage{xcolor}
	basicstyle=\footnotesize, % the size of the fonts that are used for the code
	breakatwhitespace=false, % sets if automatic breaks should only happen at whitespace
	breaklines=true, % sets automatic line breaking
	captionpos=b, % sets the caption-position to bottom
	commentstyle=\color{mygreen}, % comment style
	deletekeywords={...}, % if you want to delete keywords from the given language
	escapeinside={\%*}{*)}, % if you want to add LaTeX within your code
	extendedchars=true, % lets you use non-ASCII characters; for 8-bits encodings only, does not work with UTF-8
	frame=single, % adds a frame around the code
	keepspaces=true, % keeps spaces in text, useful for keeping indentation of code (possibly needs columns=flexible)
	keywordstyle=\color{blue}, % keyword style
	% language=Octave, % the language of the code
	morekeywords={*,...}, % if you want to add more keywords to the set
	%numbers=left, % where to put the line-numbers; possible values are (none, left, right)
	%numbersep=5pt, % how far the line-numbers are from the code
	%numberstyle=\tiny\color{mygray}, % the style that is used for the line-numbers
	rulecolor=\color{black}, % if not set, the frame-color may be changed on line-breaks within not-black text (e.g. comments (green here))
	showspaces=false, % show spaces everywhere adding particular underscores; it overrides 'showstringspaces'
	showstringspaces=false, % underline spaces within strings only
	showtabs=false, % show tabs within strings adding particular underscores
	stepnumber=1, % the step between two line-numbers. If it's 1, each line will be numbered
	stringstyle=\color{mymauve}, % string literal style
	tabsize=2, % sets default tabsize to 2 spaces
	title=\lstname % show the filename of files included with \lstinputlisting; also try caption instead of title
}
%END of listing package%

\definecolor{darkgray}{rgb}{.4,.4,.4}
\definecolor{purple}{rgb}{0.65, 0.12, 0.82}

%define Javascript language
\lstdefinelanguage{JavaScript}{
	keywords={typeof, new, true, false, catch, function, return, null, catch, switch, var, if, in, while, do, else, case, break},
	keywordstyle=\color{black}\bfseries,
	ndkeywords={class, export, boolean, throw, implements, import, this},
	ndkeywordstyle=\color{black}\bfseries,
	identifierstyle=\color{black},
	sensitive=false,
	comment=[l]{//},
	morecomment=[s]{/*}{*/},
	commentstyle=\color{black}\ttfamily,
	stringstyle=\color{black}\ttfamily,
	morestring=[b]',
	morestring=[b]"
}

\lstset{
	language=JavaScript,
	extendedchars=true,
	basicstyle=\footnotesize\ttfamily,
	showstringspaces=false,
	showspaces=false,
	tabsize=2,
	breaklines=true, 
	showtabs=false,
	captionpos=b
}

\definecolor{delim}{RGB}{20,105,176}
\definecolor{numb}{RGB}{106, 109, 32}
\definecolor{string}{rgb}{0.64,0.08,0.08}

\lstdefinelanguage{json}{
    numbers=left,
    numberstyle=\small,
    frame=single,
    rulecolor=\color{black},
    showspaces=false,
    showtabs=false,
    breaklines=true,
    postbreak=\raisebox{0ex}[0ex][0ex]{\ensuremath{\color{gray}\hookrightarrow\space}},
    breakatwhitespace=true,
    basicstyle=\ttfamily\small,
    upquote=true,
    morestring=[b]",
    stringstyle=\color{string},
    literate=
     *{0}{{{\color{numb}0}}}{1}
      {1}{{{\color{numb}1}}}{1}
      {2}{{{\color{numb}2}}}{1}
      {3}{{{\color{numb}3}}}{1}
      {4}{{{\color{numb}4}}}{1}
      {5}{{{\color{numb}5}}}{1}
      {6}{{{\color{numb}6}}}{1}
      {7}{{{\color{numb}7}}}{1}
      {8}{{{\color{numb}8}}}{1}
      {9}{{{\color{numb}9}}}{1}
      {\{}{{{\color{delim}{\{}}}}{1}
      {\}}{{{\color{delim}{\}}}}}{1}
      {[}{{{\color{delim}{[}}}}{1}
      {]}{{{\color{delim}{]}}}}{1},
}

\graphicspath{{img/}}
\RequirePackage[
  style=gost-numeric,
  language=auto,
  autolang=other,
  sorting=none,
]{biblatex}

\usepackage{csquotes}
\addbibresource{bib.bib}
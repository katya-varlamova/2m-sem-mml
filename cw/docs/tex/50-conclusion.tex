\chapter*{Заключение}
\addcontentsline{toc}{chapter}{ЗАКЛЮЧЕНИЕ}
В результате работы была решена задача регрессии, состоящая в определении выпускного балла студентов с помощью информации об их образе жизни, на данных о студентах двух португальских школ, взятых с kaggle \cite{bib:kaggle}. Таким образом, цель работы была достигнута.

Для достижения поставленной цели были решены следующие задачи:

\begin{itemize}
    \item описан и визуализирован набор данных;
    \item проанализированы существующие модели регрессии для решения задачи и выбрана наиболее подходящая, а также выбран функционал качества модели;
    \item описан алгоритм предобработки данных;
    \item описан общий алгоритм работы ПО, осуществляющего решение задачи регрессии;
    \item разработано ПО, решающее задачу регрессии;
    \item выбраны гиперпараметры модели, с которыми модель работает наилучшим образом с точки зрения функционала качества, оценена обобщающую способность;
    \item проведено исследование ПО с целью сравнения полученной модели с другими моделями с точки зрения функционала качества.
\end{itemize}
В ходе выполнения экспериментально-исследовательской части было установлено, что модель градиентного бустинга работает точнее остальных с точки зрения функционала качества.